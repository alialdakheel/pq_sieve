\documentclass{article}
\usepackage[utf8]{inputenc}
\usepackage{amssymb, amsthm}
\usepackage[normalem]{ulem} 
\usepackage{float} 
\usepackage{tikz} 
\usepackage{algpseudocode}
\usepackage{algorithm}
\begin{document}

% \section{Technical Details}
% In this section, we explain in detail the methodology to find a formula for $p+q$. Let $n=pq$ and $r=p+q$. We first define general forms for $p$ and $q$. Then we consider all the possible forms that $n$ can have which are all the different combinations of the product $pq$. After that we explain how to find $r$ for one form of $n$. We omit the details for the rest of the forms of $n$ which follow the same methodology, however, we provide the rest of the formulas in the Appendix for reference.
% %%%%%%%%%%%%%%%%%%%%
% \subsection{Divisibility by 8}
% %%%%%%%%%%%%%%%%%%%%

% \begin{proposition}\label{prop1}
% Let $p$ and $q$ be distinct odd primes, $n = pq$, and $r = p + q$ then:
% \begin{equation*}
% 8 \mid n+1 \text{ if and only if } 8 \mid r
% \end{equation*}
% \end{proposition}

% \begin{proof}
% Let $p$ and $q$ be two distinct odd primes, then they have one of the forms:
% \begin{enumerate}
%     \item $8k + 1$
%     \item $8k + 3$
%     \item $8k + 5$
%     \item $8k + 7$
% \end{enumerate}
% then the possible forms of $n$:
% \begin{enumerate}
% \item $\begin{aligned}[t]
%     (8k + 1)(8m + 1) &= 64km + 8(k + m) + 1 \\
%     &= 8k' + 1
%     \end{aligned}$
% \item $\begin{aligned}[t]
%     (8k + 1)(8m + 3) &= 64km + 8(3k + m) + 3 \\
%     &= 8k' + 3
%     \end{aligned}$
% \item $\begin{aligned}[t]
%     (8k + 1)(8m + 5) &= 64km + 8(5k + m) + 5 \\
%     &= 8k' + 5
%     \end{aligned}$

% \item $\begin{aligned}[t]
%     (8k + 1)(8m + 7) &= 64km + 8(7k + m) + 7 \\
%     &= 8k' + 7
%     \end{aligned}$
    
% \item $\begin{aligned}[t]
%     (8k + 3)(8m + 3) &= 64km + 24(k + m) + 9 \\
%     &= 8k' + 1
%     \end{aligned}$

% \item $\begin{aligned}[t]
%     (8k + 3)(8m + 5) &= 64km + 8(5k + 3m) + 15 \\
%     &= 8k' + 7
%     \end{aligned}$
    
%     \item $\begin{aligned}[t]
%     (8k + 3)(8m + 7) &= 64km + 8(7k + 5m) + 21 \\
%     &= 8k' + 5
%     \end{aligned}$
    
%     \item $\begin{aligned}[t]
%     (8k + 5)(8m + 5) &= 64km + 40(k + m) + 25 \\
%     &= 8k' + 1
%     \end{aligned}$
    
%     \item $\begin{aligned}[t]
%     (8k + 5)(8m + 7) &= 64km + 8(7k + 5m) + 35 \\
%     &= 8k' + 3
%     \end{aligned}$
    
%     \item $\begin{aligned}[t]
%     (8k + 7)(8m + 7) &= 64km + 56(k + m) + 49 \\
%     &= 8k' + 1
%     \end{aligned}$
% \end{enumerate}

% and the possible forms of $r$ will respectively be:
% \renewcommand{\labelenumi}{\alph{enumi})}
% \begin{enumerate}
% \item $\begin{aligned}[t]
%     (8k + 1) + (8m + 1) &= 8(k + m) + 2 \\
%     &= 8k' + 2
%     \end{aligned}$
% \item $\begin{aligned}[t]
%     (8k + 1) + (8m + 3) &= 8(k + m) + 4 \\
%     &= 8k'+4
%     \end{aligned}$
% \item $\begin{aligned}[t]
%     (8k + 1) + (8m + 5) &= 8(k + m) + 6 \\
%     &= 8k'+ 6
%     \end{aligned}$
%  \item $\begin{aligned}[t]
%     (8k + 1) + (8m + 7) &= 8(k + m+1) \\
%     &= 8k'
%     \end{aligned}$
%      \item $\begin{aligned}[t]
%     (8k + 3) + (8m + 3) &= 8(k + m)+6 \\
%     &= 8k'+6
%     \end{aligned}$
%  \item $\begin{aligned}[t]
%     (8k + 3) + (8m + 5) &= 8(k + m+1) \\
%     &= 8k'
%     \end{aligned}$
    
%      \item $\begin{aligned}[t]
%     (8k + 3) + (8m + 7) &= 8(k + m)+10 \\
%     &= 8k'+2
%     \end{aligned}$
    
%      \item $\begin{aligned}[t]
%     (8k + 5) + (8m + 5) &= 8(k + m)+10 \\
%     &= 8k'+2
%     \end{aligned}$

%  \item $\begin{aligned}[t]
%     (8k + 5) + (8m + 7) &= 8(k + m)+12 \\
%     &= 8k'+4
%     \end{aligned}$

%  \item $\begin{aligned}[t]
%     (8k + 7) + (8m + 7) &= 8(k + m)+14 \\
%     &= 8k'+6
%     \end{aligned}$




% \end{enumerate}
% Here $n+1$ is divisible by $8$ for the fourth and sixth forms, and similarly the corresponding forms of $r$ are also divisible by $8$. The other forms for $n+1$ and their corresponding forms for $r$ are not divisible by $8$. Therefore, we conclude that ${8 \mid n+1 \textit{ if and only if } 8 \mid r}$
% \end{proof}
% We also extend previous results in a similar fashion applicable on semi-primes equal to $3$ or $7 \mod 10$ as shown in the following tables.
% \begin{table}[H]
% \centering
% \caption{$n=3 \mod 10$}
% \label{table:21}

% \begin{tabular}{|l|l|}
% \hline
% \multicolumn{2}{|c|}{Tens odd} \\ \hline
% $3 \mid n+1$ & $3 \nmid n+1$         \\ \hline
% $r=60k+54$ & $r=60k+14$        \\ \hline
% $r=60k+6$  & $r=60k+34$        \\ \hline
%              & $r=60k+26$       \\ \hline

% & $r=60k+46$         \\ \hline
% \end{tabular}

% \bigskip

% \begin{tabular}{|l|l|l|l|}
% \hline
% \multicolumn{4}{|c|}{Tens even} \\ \hline
% \multicolumn{2}{|c|}{$8 \mid n+1$} & \multicolumn{2}{|c|}{$8 \nmid n+1$} \\ \hline
% $3 \mid n+1$ & $3 \nmid n+1$  & $3 \mid n+1$ & $3 \nmid n+1$  \\ \hline
% $r=120k+24$  & $r=120k+64$ & $r=120k+84$   & $r=120k+4$ \\ \hline
% $r=120k+96$  & $r=120k+104$ & $r=120k+36$   & $r=120k+44$ \\ \hline
%              & $r=120k+16$ &               & $r=120k+76$ \\ \hline
%              & $r=120k+56$ &               & $r=120k+116$  \\ \hline
% \end{tabular}

% \end{table}

% \begin{table}[H]
% \centering
% \caption{$n=7 \mod 10$}
% \label{table:22}

% \begin{tabular}{|l|l|}
% \hline
% \multicolumn{2}{|c|}{Tens odd} \\ \hline
% $3 \mid n+1$ & $3 \nmid n+1$         \\ \hline
% $r=60k+18$ & $r=60k+38$       \\ \hline
% $r=60k+42$ & $r=60k+58$       \\ \hline
%              & $r=60k+2$        \\ \hline
%              & $r=60k+22$         \\ \hline
% \end{tabular}

% \bigskip

% \begin{tabular}{|l|l|l|l|}
% \hline
% \multicolumn{4}{|c|}{Tens even} \\ \hline
% \multicolumn{2}{|c|}{$8 \mid n+1$} & \multicolumn{2}{|c|}{$8 \nmid n+1$} \\ \hline
% $3 \mid n+1$ & $3 \nmid n+1$  & $3 \mid n+1$ & $3 \nmid n+1$  \\ \hline
% $r=120k+48$  & $r=120k+8$ & $r=120k+108$   & $r=120k+68$ \\ \hline
% $r=120k+72$  & $r=120k+88$ & $r=120k+12$   & $r=120k+28$ \\ \hline
%              & $r=120k+32$ &               & $r=120k+92$ \\ \hline
%              & $r=120k+112$ &               & $r=120k+52$  \\ \hline
% \end{tabular}

% \end{table}
%%%%%%%%%%%%%%%%%%%%%%%%%%%%%%%%%%%%%%%%%%%%%%%%%%%%%%%%%%%%%%%%%%%%%%%%%%%%%%%%%%%%%%%%%%%%%%
\section{Product and Sum of Two Primes}\label{sec:1}
\textcolor{blue}{
%\subsection{Using prime fields}\label{ssec:2}
We noticed in~\cite{Mohammed2017}, that $2,3 \mid n+1$ if and only if $2,3 \mid r$ for any prime greater than 3, this is not true.
\begin{proposition}\label{prop:a}
Let $h$ be an odd prime greater than 3,
\begin{equation*}
\textit{if $h \mid n+1$ , then it is not necessary that } h \mid r
\end{equation*}
\end{proposition}
\begin{proof}
~\\
% \hfill \newline
% \noindent
Let $h$ be prime greater than $3$, then $h-1 = 2\frac{h-1}{2} $ where $\frac{h-1}{2} \geq 2$.
Let $n = pq$, where $p$ and $q$ are odd primes such that:\\
$p = 2 \mod h$ and $q = \frac{h-1}{2} \mod h$, then $n = pq = 2 \frac{h-1}{2} = -1 \mod h$
, so $n+1 = 0 \mod h$, then $h \mid n+1$
but $r \mod h = p + q \mod h = 2 - 2^{-1} = 2^{-1}(4-1)=2^{-1}3 \neq 0  \mod h$ , so $h \nmid r$.
\end{proof}
Let $h=7$, $n=3639$, then $7 \mid n+1$ and $r \mod 7 = (p+q) \mod 7 = 5$, so $7 \nmid r$. in fact, if $n+1 = 0 \mod 7$, then $r \mod 7$ could have one of the values: $0,2$ or $5$.
In general, there is a relation between $n \mod h$ and $r \mod h$.
for example:
let $h=7$, then the following table contains the possible values of $r \mod 7$, for each $n \mod 7 \neq 1$.
\begin{table}[H]
\centering
\caption{$n \mod 7$}
\label{table:a}
\begin{tabular}{|c|c|}
\hline
$n \mod 7$          & $r \mod 7$		\\ \hline
1 & 1, 2, 5, 6		\\ \hline
2 & 1, 3, 4, 6		\\ \hline
3 & 0, 3, 4		\\ \hline
4 & 2, 3, 4, 5		\\ \hline
5 & 0, 1, 6		\\ \hline
6 & 0, 2, 5		\\ \hline
\end{tabular}
\end{table}
As shown in table~\ref{table:a} if $n \mod 7 \neq 0$, then $r \mod 7$ has either 3 or 4 values. In general, for any odd prime $h$, if $n \mod h \neq 0$, then $r \mod h$ has $\frac{h-1}{2}$ or $\frac{h+1}{2}$ different values.\\
\begin{lemma}\label{lemma:a}
Let $h$ be prime and $a$,$b$,$c$,$d \in \mathbb{Z}_h^*$ be distinct mod $h$
\begin{equation*}
\text{If } ab = cd \mod h, \text{ then } a+b \neq c+d \mod h
\end{equation*}
\end{lemma}
\begin{proof}
~\\
% \hfill \newline
% \noindent
Let $a$,$b$,$c$,$d \in \mathbb{Z}_h^*$ and distinct with $ab = cd \mod h$.\\
Suppose $a+b = c+d \mod h$ 
then $a = c+d - b \mod h $
$$\implies (c+d -b)b = cd \mod h $$
$$\implies cb + db -b^2 = cd \mod h $$
$$\implies db - b ^2 = cd -cb \mod h$$
$$\implies (d-b)b = c(d-b) \mod h$$
$$\implies (d-b)b = (d-b)c \mod h$$
Since $d-b \neq 0 \mod h$ which means $b= c \mod h$. 
However, this contradicts the assumption, So $a+b \neq c+d \mod h$
\end{proof}
\
\begin{proposition}\label{prop:b}
Let $n = pq$ and $r = p+q$ where $p$, $q$, $h$ are distinct odd primes With $n \neq 0 \bmod h$ then:
\begin{equation*}
r \bmod h \text{ has } \frac{h-1}{2} \text{ or } \frac{h+1}{2} \text{ different possible values.}
\end{equation*}
\end{proposition}
\
\begin{proof}
~\\
% \hfill \newline
% \noindent
Let $ a \in \mathbb{Z}_h$. $a$ is a quadratic residue modulo $h$ if $\exists x \in \mathbb{Z}_h^*$ such that $a = x^2 \mod h$. If $a$ isn't a quadratic residue then for any $b \in \mathbb{Z}_h^*$, $\exists c \neq b \in \mathbb{Z}_h^*$ such that $bc = a \mod h$.
$$b_1 c_1 = b_2 c_2 = ... = b_{h-1} c_{h-1} = a \mod h$$
Since the multiplication and addition is commutative, the above becomes:
$$b_1 c_1 = b_2 c_2 = ... = b_{\frac{h-1}{2}} c_{\frac{h-1}{2}} = a \mod h$$
So by lemma~\ref{lemma:a}: $b_i + c_i \neq b_j + c_j$ if $i \neq j$ then we have $\frac{h-1}{2}$ different $r \mod h$.\\
\\
Now, suppose a is a quadratic residue then $\exists x \in \mathbb{Z}_h^*$ , such that $a = x^2 \mod h$, 
and $a = (h-x)^2 \mod h$ so we have 
$$b_1 c_1 = b_2 c_2 = ... = b_{\frac{h-3}{2}} c_{\frac{h-3}{2}} = b_{\frac{h-1}{2}}^2 =  b_{\frac{h+1}{2}}^2 = a \mod h$$
So we have $\frac{h+1}{2}$ different $r \mod h$
\
% such that $s \neq a^2 \mod h$\\
% So, for any $b \in \mathbb{Z}_h$ there exists a distinct number $c \in \mathbb{Z}_h$ such that $bc = s$\\
% So we have:\\
% $$b_1 c_1 = b_2 c_2 = ... = b_{n-1} c_{n-1} = s \mod h$$
% Since $bc =cb$, for any $b, c \in \mathbb{Z}_h$, we will consider $b_i c_i$ and $ b_j c_j$ as one multiplication if $b_i = c_j$ and $b_j = c_i$\\
% Now we have different $\frac{h-1}{2}$ pairs and their multiplication is equal to $s$.\\
% $$b_1 c_1 = b_2 c_2 = ... = b_{\frac{h-1}{2}} c_{\frac{h-1}{2}}$$
% Now, we want to prove that if $a$, $b$, $c$, $d$ distict and $ab = cd \mod h$, then $a+b \neq c+d \mod h$ \\
% Let $a$,$b$,$c$,$d$ be distinct and $ab = cd \mod h$\\
% Suppose $a+b = c+d \mod h$ then $a = c+d - b \mod h$ so $(c+d-b)b = cd \mod h $\\
% $$\implies cb +db - b^2 = cd \mod h$$ 
% $$\implies db - b^2 = cd - cb \mod h$$
% $$\implies (d - b) b = c(d - b) \mod h$$
% Since $d \neq b$, so $b=c \mod h$ which is a contradiction. \\
% So if $ab = cd = s \mod h$, then $a+b \neq c+d \mod h$ \\
% Then we have: $\frac{h-1}{2}$ different $r$\\
% Now, suppose $s = a^2 \mod h$ for some $a \in \mathbb{Z}_h$ , then $s = (h-a)^2 \mod h$,
% but for two pairs $a_i = b_i$ we have:
% $$a^2 = (h - a)^2 = s \mod h$$ 
% $$2a \neq -2a \mod h$$
% So if $s = a ^2 \mod h$ , $a \neq 0 \mod h$, then we have $\frac{h+1}{2}$ different $r$.
\end{proof}
			The ideas described above hint on the algorithm's dependence on pre-computation, 
			and there are several aspects to consider.
			How far can we pre-process? and whether it is useful to go to such lengths. 
            To calculate $r \bmod h$ for all $n \bmod h$, one can find the Cayley tables for multiplication and addition then save the unique values in a manner similar to that of table~\ref{table:a}. 
		\subsubsection{Exhaustive Listing}
			It is also possible to use a more efficient listing algorithm.
			We search all useful combinations with second terms greater than or equal to the first term 
			then store the possible values as lists 
			with their corresponding $n \bmod p$ as keys to those lists as in~\ref{alg:method1}
			\begin{algorithm}[H]
				\caption{Listing}
				\label{alg:method1}
				\begin{algorithmic}[1]
					\Require $h \gets \text{prime} \geq 7$
					\For{ $i$ in $[1,p-1]$ } 
						\For{ $j$ in $[i,p-1]$ } 
							\State $permval[ij+1] \gets (i+j) \mod p$
						\EndFor
					\EndFor
				\end{algorithmic}
			\end{algorithm}
		\subsubsection{`$r$' for a Specific `$n \bmod h$'}
			Let $x= n \bmod h$ and $r = i + i^{-1} x$ where $i \in \mathbb{Z}_h^*$
			\begin{proof}
				$$ij = x \bmod h$$
				$$j = x i^{-1} \bmod h$$
				$$r = i + j = x i^{-1} \bmod h$$
				Notice if $r \bmod h $ is a possible value, then $(-r) \bmod h$ is also a possible value.
				Since $-r = -( x i^{-1} + i )$
				$= -x i^{-1} - i$ 
				$= x (-i)^{-1} -i$
			\end{proof}
		\subsubsection{Using Inverse}
			\begin{algorithm}
				\caption{Using $n \bmod h$ multiplied by $i^{-1}$}
				\label{alg:method2}
				\begin{algorithmic}[1]
					\Require $h$, $n$
					%\REQUIRE $n \gets \text{Integer to be factored}$
					\Ensure $h \geq 7$
					\For{$i$ in $[1,h-1]$} 
                    	\If{$i \text{ in } ni\_list$}
							\State continue
                        \EndIf
                    	\State $ni \gets n * i^{-1} \bmod h$
                        \State append $ni$ to $ni\_list$
						\State $r \gets ( i + ni ) \bmod h$
						\If{$r \text{ in } r\_list$}
							\State continue
						\EndIf
						\State append $r$ to $r\_list$
						\If {$r \neq 0$}
							\State append $(h-r)$ to $r\_list$
						\EndIf
					\EndFor
					%\PROCEDURE{myPro}{()a}
					%\STATE { hello  }
					%\ENDPROCEDURE
				\end{algorithmic}
			\end{algorithm}
}

\section{Divisibility by 8}\label{ssec:1}
%%%%%%%%%%%%%%%%%%%%

\begin{proposition}\label{prop1}
Let $p$ and $q$ be distinct odd primes, $n = pq$, and $r = p + q$ then:
\begin{equation*}
8 \mid n+1 \text{ if and only if } 8 \mid r
\end{equation*}
\end{proposition}
\begin{proof}
Let $p$ and $q$ be two distinct odd primes, then they have one of the forms:
\begin{enumerate}
    \item $8k + 1$
    \item $8k + 3$
    \item $8k + 5$
    \item $8k + 7$
\end{enumerate}
then the possible forms of $n$:
\begin{enumerate}
\item $\begin{aligned}[t]
    (8k + 1)(8m + 1) &= 64km + 8(k + m) + 1 \\
    &= 8k' + 1
    \end{aligned}$
\item $\begin{aligned}[t]
    (8k + 1)(8m + 3) &= 64km + 8(3k + m) + 3 \\
    &= 8k' + 3
    \end{aligned}$
\item $\begin{aligned}[t]
    (8k + 1)(8m + 5) &= 64km + 8(5k + m) + 5 \\
    &= 8k' + 5
    \end{aligned}$
\item $\begin{aligned}[t]
    (8k + 1)(8m + 7) &= 64km + 8(7k + m) + 7 \\
    &= 8k' + 7
    \end{aligned}$
\item $\begin{aligned}[t]
    (8k + 3)(8m + 3) &= 64km + 24(k + m) + 9 \\
    &= 8k' + 1
    \end{aligned}$
\item $\begin{aligned}[t]
    (8k + 3)(8m + 5) &= 64km + 8(5k + 3m) + 15 \\
    &= 8k' + 7
    \end{aligned}$
    \item $\begin{aligned}[t]
    (8k + 3)(8m + 7) &= 64km + 8(7k + 5m) + 21 \\
    &= 8k' + 5
    \end{aligned}$
    \item $\begin{aligned}[t]
    (8k + 5)(8m + 5) &= 64km + 40(k + m) + 25 \\
    &= 8k' + 1
    \end{aligned}$
    \item $\begin{aligned}[t]
    (8k + 5)(8m + 7) &= 64km + 8(7k + 5m) + 35 \\
    &= 8k' + 3
    \end{aligned}$
    \item $\begin{aligned}[t]
    (8k + 7)(8m + 7) &= 64km + 56(k + m) + 49 \\
    &= 8k' + 1
    \end{aligned}$
\end{enumerate}
and the possible forms of $r$ will respectively be:
\renewcommand{\labelenumi}{\alph{enumi})}
\begin{enumerate}
\item $\begin{aligned}[t]
    (8k + 1) + (8m + 1) &= 8(k + m) + 2 \\
    &= 8k' + 2
    \end{aligned}$
\item $\begin{aligned}[t]
    (8k + 1) + (8m + 3) &= 8(k + m) + 4 \\
    &= 8k'+4
    \end{aligned}$
\item $\begin{aligned}[t]
    (8k + 1) + (8m + 5) &= 8(k + m) + 6 \\
    &= 8k'+ 6
    \end{aligned}$
 \item $\begin{aligned}[t]
    (8k + 1) + (8m + 7) &= 8(k + m+1) \\
    &= 8k'
    \end{aligned}$
     \item $\begin{aligned}[t]
    (8k + 3) + (8m + 3) &= 8(k + m)+6 \\
    &= 8k'+6
    \end{aligned}$
 \item $\begin{aligned}[t]
    (8k + 3) + (8m + 5) &= 8(k + m+1) \\
    &= 8k'
    \end{aligned}$
     \item $\begin{aligned}[t]
    (8k + 3) + (8m + 7) &= 8(k + m)+10 \\
    &= 8k'+2
    \end{aligned}$
     \item $\begin{aligned}[t]
    (8k + 5) + (8m + 5) &= 8(k + m)+10 \\
    &= 8k'+2
    \end{aligned}$
 \item $\begin{aligned}[t]
    (8k + 5) + (8m + 7) &= 8(k + m)+12 \\
    &= 8k'+4
    \end{aligned}$
 \item $\begin{aligned}[t]
    (8k + 7) + (8m + 7) &= 8(k + m)+14 \\
    &= 8k'+6
    \end{aligned}$
\end{enumerate}
Here $n+1$ is divisible by $8$ for the fourth and sixth forms, and similarly the corresponding forms of $r$ are also divisible by $8$. The other forms for $n+1$ and their corresponding forms for $r$ are not divisible by $8$. Therefore, we conclude that ${8 \mid n+1 \textit{ if and only if } 8 \mid r}$
\end{proof}
We also extend previous results in a similar fashion applicable on semi-primes equal to $3$ or $7 \mod 10$ as shown in the following tables.

\begin{table}[H]
\centering
\caption{$n=3 \mod 10$}
\label{table:21}

\begin{tabular}{|l|l|}
\hline
\multicolumn{2}{|c|}{Tens odd} \\ \hline
$3 \mid n+1$ & $3 \nmid n+1$         \\ \hline
$r=60k+54$ & $r=60k+14$        \\ \hline
$r=60k+6$  & $r=60k+34$        \\ \hline
             & $r=60k+26$       \\ \hline

& $r=60k+46$         \\ \hline
\end{tabular}

\bigskip

\begin{tabular}{|l|l|l|l|}
\hline
\multicolumn{4}{|c|}{Tens even} \\ \hline
\multicolumn{2}{|c|}{$8 \mid n+1$} & \multicolumn{2}{|c|}{$8 \nmid n+1$} \\ \hline
$3 \mid n+1$ & $3 \nmid n+1$  & $3 \mid n+1$ & $3 \nmid n+1$  \\ \hline
$r=120k+24$  & $r=120k+64$ & $r=120k+84$   & $r=120k+4$ \\ \hline
$r=120k+96$  & $r=120k+104$ & $r=120k+36$   & $r=120k+44$ \\ \hline
             & $r=120k+16$ &               & $r=120k+76$ \\ \hline
             & $r=120k+56$ &               & $r=120k+116$  \\ \hline
\end{tabular}
\end{table}
\begin{table}[H]
\centering
\caption{$n=7 \mod 10$}
\label{table:22}
\begin{tabular}{|l|l|}
\hline
\multicolumn{2}{|c|}{Tens odd} \\ \hline
$3 \mid n+1$ & $3 \nmid n+1$         \\ \hline
$r=60k+18$ & $r=60k+38$       \\ \hline
$r=60k+42$ & $r=60k+58$       \\ \hline
             & $r=60k+2$        \\ \hline
             & $r=60k+22$         \\ \hline
\end{tabular}

\bigskip

\begin{tabular}{|l|l|l|l|}
\hline
\multicolumn{4}{|c|}{Tens even} \\ \hline
\multicolumn{2}{|c|}{$8 \mid n+1$} & \multicolumn{2}{|c|}{$8 \nmid n+1$} \\ \hline
$3 \mid n+1$ & $3 \nmid n+1$  & $3 \mid n+1$ & $3 \nmid n+1$  \\ \hline
$r=120k+48$  & $r=120k+8$ & $r=120k+108$   & $r=120k+68$ \\ \hline
$r=120k+72$  & $r=120k+88$ & $r=120k+12$   & $r=120k+28$ \\ \hline
             & $r=120k+32$ &               & $r=120k+92$ \\ \hline
             & $r=120k+112$ &               & $r=120k+52$  \\ \hline
\end{tabular}
\end{table}
\section{Formulas and Prime Fields}\label{sec:2}
\textcolor{ForestGreen}{
\begin{proposition}\label{prop:c}
For any formula $r(k) = a k + b$ in \cite{Mohammed2017}, and a prime $h \geq 7$.
\begin{equation*}
r(k) = r(m) \bmod h \text{ if and only if } k = m \bmod h
\end{equation*}
\end{proposition}
\begin{proof}
% \hfill \newline
% \noindent
~\\
Let $r(k) = ak +b \mod h$, with h a prime, and $a = 2^{\alpha_1} 3^{\alpha_2} 5^{\alpha_3}$, then:\\
$$r(k) = r(m) \mod h$$
$$\iff ak+b = am+b \mod h$$
$$\iff ak = am \mod h$$
$$\iff k=m \mod h$$
%old proof
% For $r_1 = a k_1 +b$ and $r_2 = a k_2 + b$\\
% Let $r_1 = r_2$, then 
% $ak_1 + b = ak_2 + b \mod h$ 
% $$\implies a k_1 = a k_2 \mod h$$ 
% $$\implies k_1 = k_2 \mod h$$
\end{proof}
% \hfill \newline
% \noindent
From proposition~\ref{prop:c} we have:
\begin{enumerate}
\item $r(k) \bmod h$, $r(k+1) \bmod h$, ..., $r(k+h-1) \bmod h$ are distinct, so 
\begin{equation*}
\mathbb{Z}_h = \{r(k) \bmod h, ..., r(k+h-1)\}
\end{equation*}
then $\frac{h-1}{2}$ or $\frac{h+1}{2}$ of $r(k)$, $r(k+1)$, ..., $r(k+h-1)$ cannot be equal to $p+q$. In the previous example $n= 3639$ and $n \bmod 7 = 6$, the possible values of $r \bmod 7$ are 0,2, and 5 as shown in table~\ref{table:a}. So, if $r(k) \bmod 7 = 3$, then $r(k) \neq p+q$
\item If $r(k) \bmod h \neq$ one of the possible values of $r \bmod h$, then 
$r(k+\gamma h) \neq$ one of the possible values of $r \bmod h$. If $r(k) \bmod 7 =3 $, then by proposition~\ref{prop:c}, $r(k+7\gamma) \bmod 7 = 3$. So, $r(k+7\gamma) \neq p+q$.
\end{enumerate}
% Each evaluation of any of the formulas namely $[r(k), r(k+1), ..., r(k+h-1)]$
% is in a different congruence class modulo a prime `$h$', combining this with proposition \ref{prop:b} we describe two results:
% %$\frac{h-1}{2}$ or $\frac{h+1}{2}$ of $r \mod h$ doesn't satisfy $r = p+q$, So we have two things:
% \begin{enumerate}
% \item For $r(k)$, $r(k+1)$, ..., $r(k+(h-1))$
%  a solution exists only in $\frac{h-1}{2}$ or $\frac{h+1}{2}$ of the congruence class. So, we need to check only half of the reduced search space. 
%  In other words, if `$r(k)$' is not in one of the congruence classes corresponding to `$n$', then $r(k) \neq p+q$,
% \item $r(k+\gamma h)$ is not of interest, If $r(k)$ does not satisfy the previous condition. 
% So we only need to find ($r(k) \bmod h$) on the interval [k,k+h), which results in a significant decrease in the number of `$r(k) \mod h$' computations, and an overall number of tries $\approx \frac{\mid p-q \mid}{2}$
% \end{enumerate}
% Example comparison with paper 1
% \section{Previous Work}
The method suggested in~\cite{Mohammed2017} is to compute the starting points for a number of predefined formulas $r_i(k)$ using $\lceil 2\sqrt{n} \rceil$, the formulas are chosen depending on certain characteristics of `$n$'. After which, the condition 
\begin{equation}\label{eq:condition}
r_i(k) = p+q \iff \sqrt{r^2-4n} \text{ is a positive integer}
\end{equation}
, is checked while iteratively incrementing $k$.\\ 
~\\
%
%
% \subsection{Example (previous method)}
%  In this section, we give an example on how we can find $r=p+q$ given $n$, and subsequently $p$ and $q$ using equation \ref{eq:1}. Notice that $p+q > 2\sqrt{n}$ providing a starting point for each formula.
% 
% We start by looking at the value $n$ and find the table of formulas corresponding to the form of $n$. After that, we use the formulas to search for $r$. We find $r$ when the value $\sqrt{r^2 - 4n}$ is equal to an integer (see Appendix A).
%
%Given $n=9177506999$ (constructed from $p=104729$ and $q=87631$)
For example, given $n=234948664218045611$ (computed from $p=925106617$ and $q=253969283$) we choose the table 6 from~\cite{Mohammed2017}. Compute the starting points for each of the three formulas $r_1(k_{1,0})$, $r_2(k_{1,0})$, $r_3(k_{3,0})$ using $2\sqrt{n} \approx 969430068$. Check the  condition $r_i(k) = p+q \iff \sqrt{r^2-4n}$ is a positive integer iteratively while incrementing $k$. \\
% \begin{enumerate}\setlength\itemsep{1em}
%     \item $n=11 \mod 100$ (choose table 10 in paper 1).
%     \item We choose the corresponding r(k) formulas and compute starting points with  $2\sqrt{n} \approx 969430068$: \\
%     $r_1(k) = 600k_1+12$,	$k_{1,0} = 1615716$ \\
%     $r_2(k) = 600k_2+588$,	$k_{2,0} = 1615717$ \\ 
%     $r_3(k) = 120k_3$+60,	$k_{3,0} = 8078584$ 
% \item We evaluate $r_1, r_2, r_3$ starting with the above values for $k_{1,0}, k_{2,0}, k_{3,0}$ respectively, and continue to increment the $k$'s until $\sqrt{r^2 - 4n}$ is equal to an integer.
%     $k_{3,1} = k_{3,0} + 1 = 1624$ \\
%     Now we compute $r$ in the following way: \\
%     $r = 120(k_{3,1}) = 120(1624)=194880$
%     \item $p-q = \sqrt{r^2-4n} = \sqrt{37978214400-37905862364} = 8506$
%     \item $p = {[r+(p-q)]/2} = 203386/2 = 101693$ \\ 
%     $q = n/p = 9476465591/101693 = 93187$
% \end{enumerate}
% (Notice in this example that $k$ is small; this is due to $p-q$ being small.)
% \begin{table}[H]
% \centering
% \caption{First 7 iterations of the example}
% \label{table:b}
% \begin{tabular}{|c|c|c|c|}
% \hline
% i	& $r_1(k)$	& $r_2(k)$	& $r_3(k)$	\\ \hline
% 0	& $969430212 $	& $969430188 $	& $969430140 $	\\ \hline
% 1	& $969430812 $	& $969430788 $	& $969430260 $	\\ \hline
% 2	& $969431412 $	& $969431388 $	& $969430380 $	\\ \hline
% 3	& $969432012 $	& $969431988 $	& $969430500 $	\\ \hline
% 4	& $969432612 $	& $969432588 $	& $969430620 $	\\ \hline
% 5	& $969433212 $	& $969433188 $	& $969430740 $	\\ \hline
% 6	& $969433812 $	& $969433788 $	& $969430860 $	\\ \hline
% \end{tabular}
% \end{table}
The value for $k$ that satisfy the condition is $k_{3,1747048} = 9825632$\\
$r_3(9825632) = 1179075900$.\\
Number of tries  = $3 \times 1747048 = 5241144$\\
%     \item $p-q = \sqrt{r^2-4n} = \sqrt{1390219977960810000-939794656872182444} = 671137334$
%     \item $p = {[r+(p-q)]/2} = 1850213234/2 = 925106617$ \\ 
%     $q = n/p = 234948664218045611/925106617 = 253969283$
% \end{enumerate}
~\\
Using the results from proposition~\ref{prop:c}, one can approximately halve the number of expensive condition checks in the previous method. Computing modulo h operations h times then only checking the condition for the values in a permissible congruence class modulo h. The decrease in the number of condition checks depend on whether `$n$' is a quadratic residue modulo h which is discussed in depth later in~\ref{ssec:choose_prime}.
\subsection{Example (with $h=7$)}
Given $n=234948664218045611$ \\
% (constructed from $p=925106617$ and $q=253969283$)
%     \item $n=11 \mod 100$, so we look at table~\ref{table:2} in Appendix B.
%     \item We choose the corresponding r(k) formulas and compute starting points with  $2\sqrt{n} \approx 969430068$: \\
%     $r_1(k) = 600k_1+12$,	$k_{1,0} = 1615716$ \\
%     $r_2(k) = 600k_2+588$,	$k_{2,0} = 1615717$ \\ 
%     $r_3(k) = 120k_3$+60,	$k_{3,0} = 8078584$ 
Rather than evaluating $r_1(k), r_2(k), r_3(k)$ starting with the above values for $k_{1,0}, k_{2,0}, k_{3,0}$ respectively then continuing to increment the $k$'s until $\sqrt{r^2 - 4n}$ is equal to an integer, we use proposition~\ref{prop:a} with $ h= 7$, then compute $n \mod h = 1$ and evaluate the first h values of ($k_{1,i}, k_{2,i}, k_{3,i}$). After which, we check if they map to the possible values modulo $h$ as shown in table~\ref{table:a}, if not we discard the value.\\
\begin{table}[H]
\centering
\caption{First 7 iterations of the example}
\label{table:b}
\begin{tabular}{|c|c|c|c|}
\hline
i	& $r_1(k)$	& $r_2(k)$	& $r_3(k)$	\\ \hline
0	& $969430212 \bmod h = 2$	& $969430188 \bmod h = 6$	& \hbox{\sout{$969430140 \bmod h = 0$}}	\\ \hline
1	& \hbox{\sout{$969430812 \bmod h = 0$}}	& \hbox{\sout{$969430788 \bmod h = 4$}}	& $969430260 \bmod h = 1$	\\ \hline
2	& $969431412 \bmod h = 5$	& $969431388 \bmod h = 2$	& $969430380 \bmod h = 2$	\\ \hline
3	& \hbox{\sout{$969432012 \bmod h = 3$}}	& \hbox{\sout{$969431988 \bmod h = 0$}}	& \hbox{\sout{$969430500 \bmod h = 3$}}	\\ \hline
4	& $969432612 \bmod h = 1$	& $969432588 \bmod h = 5$	& \hbox{\sout{$969430620 \bmod h = 4$}}	\\ \hline
5	& $969433212 \bmod h = 6$	& \hbox{\sout{$969433188 \bmod h = 3$}}	& $969430740 \bmod h = 5$	\\ \hline
6	& \hbox{\sout{$969433812 \bmod h = 4$}}	& $969433788 \bmod h = 1$	& $969430860 \bmod h = 6$	\\ \hline
\end{tabular}
\end{table}
% \begin{table}[H]
% \centering
% \caption{First 7 iterations of the example}
% \label{table:b}
% \begin{tabular}{|c|c|c|c|c|}
% \hline
% i	& $k_{1,i}$, $k_{2,i}$, $k_{3,i}$ & $r_1(k)$	& $r_2(k)$	& $r_3(k)$	\\ \hline
% 0	& $1615716,1615717,8078584$ & $969430212 \bmod h = 2$	& $969430188 \bmod h = 6$	& \hbox{\sout{$969430140 \bmod h = 0$}}	\\ \hline
% 1	& $1615717$, $1615718$, $8078585$ & \hbox{\sout{$969430812 \bmod h = 0$}}	& \hbox{\sout{$969430788 \bmod h = 4$}}	& $969430260 \bmod h = 1$	\\ \hline
% 2	& $1615718$, $1615719$, $8078586$ & $969431412 \bmod h = 5$	& $969431388 \bmod h = 2$	& $969430380 \bmod h = 2$	\\ \hline
% 3	& $1615719$, $1615720$, $8078587$ & \hbox{\sout{$969432012 \bmod h = 3$}}	& \hbox{\sout{$969431988 \bmod h = 0$}}	& \hbox{\sout{$969430500 \bmod h = 3$}}	\\ \hline
% 4	& $1615720$, $1615721$, $8078588$ & $969432612 \bmod h = 1$	& $969432588 \bmod h = 5$	& \hbox{\sout{$969430620 \bmod h = 4$}}	\\ \hline
% 5	& $1615721$, $1615722$, $8078589$ & $969433212 \bmod h = 6$	& \hbox{\sout{$969433188 \bmod h = 3$}}	& $969430740 \bmod h = 5$	\\ \hline
% 6	& $1615722$, $1615723$, $8078590$ & \hbox{\sout{$969433812 \bmod h = 4$}}	& $969433788 \bmod h = 1$	& $969430860 \bmod h = 6$	\\ \hline
% \end{tabular}
% \end{table}
Discard all values $r(k+\gamma h)$ corresponding to the first h values crossed as shown in the above table. Traverse the remaining space until a value of r(k) that satisfies the condition is reached. The value for $k$ where a solution exists is $k_{3,1747048} = 9825632$.
\begin{itemize}\setlength\itemsep{1em}
\item Number of tries $= \lceil \frac{4}{7} 5241144 \rceil$
\end{itemize}
\subsection{Choosing a prime h}\label{ssec:choose_prime}
Given two primes $h_1$ and $h_2$, where $h_1 \nmid n$ and $h_2 \nmid n$. Choosing the prime that has $r \mod h$ with $\frac{h-1}{2}$ values is generally better than choosing a prime with $\frac{h+1}{2}$ values. For any two primes such that $h_2 > h_1$ and both have $\frac{h-1}{2}$ congruence classes, it is better to use $h_1$. However, if both primes $h_1$ and $h_2$ had $\frac{h+1}{2}$ values for $r \mod h$ we choose $h_2$.
}
\section{Multiple Prime Fields}\label{sec:3}
\textcolor{RawSienna}{
% \begin{proposition}\label{prop:d}
%  $<$ Adding a prime $h_2 >$
% \end{proposition}
% \begin{proof}
% \hfill \newline
% \noindent
% Let $r(k+\gamma h_1)$ be one of the possible values, where $\gamma \in \mathbb{N}$. Each value $0,1,2,...,(h_2 -1)$ is a different congruence class modulo $h_2$, multiplying by $h_1$:
% $$0,h_1,2h_1,...,(h_2 -1)$$ 
% Again each value is in a different congruence class modulo $h_2$, since $gcd(h_1, h_2) = 1$. Adding $k$:
% $$k, k+h_1, k+2h_1, ..., k+(h_2-1)h_1$$
% The values still map to unique congruence classes. So, by using $h_2 \neq h_1$, the number of conditional checks are decreased. This means that the number of tries will decrease to:
% $$ Z_2 = \left(\frac{h_2 \pm 1}{2h_2}\right) Z_1$$
% $$= \left(\frac{h_2 \pm 1}{2h_2}\right) \left(\frac{h_1 \pm 1}{2h_1}\right) Z_0$$
% $$\approx \frac{Z_0}{2^2} = \frac{Z_0}{4}$$
% \end{proof}
\begin{proposition}
Let $h_1$, $h_2$ be odd distinct primes,
$\mathbb{Z}_{h_2} = \{0,...,h_2-1\}$ and $ K = \{r(k+\gamma h_1) \bmod h_2 \mid \gamma \in \mathbb{Z}_{h_2}$\} then $K = \mathbb{Z}_{h_2}$
% \mathbb{Z}_{h_2} = \{r(k) \bmod h_2, r(k+h_1) \bmod h_2, r(k+2h_1) \bmod h_2, ..., r(k+(h_2-1)h_1) \bmod h_2\}
\end{proposition}
\begin{proof}
% \hfill \newline
% \noindent
~\\
Suppose $r(k+\gamma_1 h_1) \bmod h_2 = r(k+\gamma_2 h_1) \bmod h_2$\\
$$ \implies (a(k+\gamma_1 h_1) + b) \bmod h_2 = (a(k+\gamma_2 h_1) + b) \bmod h_2$$
$$ \implies a(k+\gamma_1 h_1) \bmod h_2 = a(k+\gamma_2 h_1) \bmod h_2$$
$$ \implies k+\gamma_1 h_1 \bmod h_2 = k+\gamma_2 h_1 \bmod h_2$$
$$ \implies \gamma_1 h_1 \bmod h_2 = \gamma_2 h_1 \bmod h_2$$
% \begin{equation*}
$$\implies \gamma_1 \bmod h_2 = \gamma_2 \bmod h_2 \text{ contradiction.}$$
% \end{equation*}
\end{proof}
By proposition~\ref{prop:c}:
$$r(k) \bmod h_1 = r(k+\gamma h_1) \bmod h_1$$
So, if $r(k) \bmod h_1$ is one of the possible values, then $r(k+ \gamma h_1)$ is also one of the possible values,
but $\frac{h_2-1}{2}$ or $\frac{h_2+1}{2}$ of $r(k), r(k+h_1), ..., r(k+(h_2-1)h_1)$
are not of the possible values modulo $h_2$
then the number of tries decreases by $\frac{h_2-1}{2h_2}$ or $\frac{h_2+1}{2h_2}$ of the number of tries.
% \begin{table}[H]
% \centering
% \caption{$n+1 \mod 11$}
% \label{table:a}
% \begin{tabular}{|c|c|}
% \hline
% $n+1 \mod 11$          & $r \mod 11$		\\ \hline
% 0 & 0, 1, 4, 7, 10		\\ \hline
% 2 & 2, 3, 4, 7, 8, 9		\\ \hline
% 3 & 0, 1, 3, 8, 10		\\ \hline
% 4 & 1, 2, 4, 7, 9, 10 		\\ \hline
% 5 & 3, 4, 5, 6, 7, 8		\\ \hline
% 6 & 1, 3, 5, 6, 8, 10	\\ \hline
% 7 & 0, 4, 5, 6, 7		\\ \hline
% 8 & 0, 2, 3, 8, 9	\\ \hline
% 9 & 0, 2, 5, 6, 9		\\ \hline
% 10 & 1, 2, 5, 6, 9, 10		\\ \hline
% \end{tabular}
% \end{table}
\begin{table}
\centering
\caption{$n \mod 13$}
\label{table:a}
\begin{tabular}{|c|c|}
\hline
$n \mod 13$          & $r \mod 13$		\\ \hline
1 & 0, 1, 2, 4, 9, 11, 12		\\ \hline
2 & 2, 3, 5, 8, 10, 11		\\ \hline
3 & 0, 3, 4, 5, 8, 9, 10 		\\ \hline
4 & 0, 2, 4, 5, 8, 9, 11		\\ \hline
5 & 2, 4, 6, 7, 9, 11	\\ \hline
6 & 1, 5, 6, 7, 8, 12		\\ \hline
7 & 1, 4, 5, 8, 9, 12	\\ \hline
8 & 3, 4, 6, 7, 9, 10		\\ \hline
9 & 0, 1, 3, 6, 7, 10, 12		\\ \hline
10 & 0, 1, 2, 6, 7, 11, 12		\\ \hline
11 & 1, 2, 3, 10, 11, 12		\\ \hline
12 & 0, 3, 5, 6, 7, 8, 10	\\ \hline
\end{tabular}
\end{table}
%Example with h= 7,11
\subsection{Example (with $h_1=7$, $h_2 = 13$)}
% \begin{itemize}%\setlength\itemsep{1em}
$n=234948664218045611$\\
% \begin{enumerate}\setlength\itemsep{1em}
%     \item $n=11 \mod 100$, so we look at table~\ref{table:2} in Appendix B.
%     \item We choose the corresponding r(k) formulas and compute starting points with  $2\sqrt{n} \approx 969430068$: \\
%     $r_1(k) = 600k_1+12$,	$k_{1,0} = 1615716$ \\
%     $r_2(k) = 600k_2+588$,	$k_{2,0} = 1615717$ \\ 
%     $r_3(k) = 120k_3$+60,	$k_{3,0} = 8078584$ 
%     \item Rather than evaluating $r_1(k), r_2(k), r_3(k)$ starting with the above values for $k_{1,0}, k_{2,0}, k_{3,0}$ respectively then continuing to increment the $k$'s until $\sqrt{r^2 - 4n}$ is equal to an integer, we use proposition~\ref{prop:a} with $ h= 7$, then compute $n+1 \mod h = 2$ and evaluate the first h values of ($k_{1,i}, k_{2,i}, k_{3,i}$). After which, we check if they map to the possible values modulo $h$ as shown in table~\ref{table:a}, if not we discard the value.\\
% \begin{tabular}{|l|l|l|l|}
% \hline
% \multicolumn{4}{|c|}{Tens even} \\ \hline
% \multicolumn{2}{|c|}{$8 \mid n+1$} & \multicolumn{2}{|c|}{$8 \nmid n+1$} \\ \hline
$n \mod h_1 = 1$\\
% \item $n+1 \mod h_2 = 10$ 
% \begin{table}[H]
% \centering
% \caption{First 7 iterations of the example}
% \label{table:b}
% \begin{tabular}{|c|c|@{}c@{}|@{}c@{}|c|@{}c@{}|@{}c@{}|c|@{}c@{}|@{}c@{}|}
% \hline
% 	& \multicolumn{3}{|c|}{$r_1(k) = 600k_1+12$} &  \multicolumn{3}{|c|}{$r_2(k) = 600k_2+588$} & \multicolumn{3}{|c|}{$r_3(k) = 120k_3+60$}	\\ \hline
% i   & $r_1(k)$	& $\bmod h_1$	& $\bmod h_2$	& $r_2(k)$	& $\bmod h_1$	& $\bmod h_2$	& $r_2(k)$	& $\bmod h_1$	& $\bmod h_2$ \\ \hline
% 0	& \hbox{\sout{$969430212$}} & 2 & 3	& $969430212$ & 6 & 1	& \hbox{\sout{$969430212$}} & 0 & 8	\\ \hline
% 1	& \hbox{\sout{$969430812$}} & 0	& 9	&	\hbox{\sout{$969430788 $}} & 4 & 7	& \hbox{\sout{$969430260$}} & 1 &	7	\\ \hline
% 2	& \hbox{\sout{$969431412$}}	& 5	& 4	& $969431388$	& 2	& 2	& $969430380$	& 2	& 6	\\ \hline
% 3	& \hbox{\sout{$969432012$}}	& 3	& 10	& \hbox{\sout{$969431988$}}	& 0	& 8	& \hbox{\sout{$969430500$}}	& 3	& 5	\\ \hline
% 4	& $969432612$	& 	1 & 5	& \hbox{\sout{$969432588$}}	& 5	& 3	& \hbox{\sout{$969430620$}}	& 4	& 4	\\ \hline
% 5	& \hbox{\sout{$969433212$}}	& 6	& 0	& \hbox{\sout{$969433188$}}	& 3	& 9	& \hbox{\sout{$969430740$}}	& 5	& 3	\\ \hline
% 6	& \hbox{\sout{$969433812$}}	& 4	& 6	& \hbox{\sout{$969433788$}}	& 1	& 4	& $969430860$	& 6	& 2	\\ \hline
% 7	& $969434412$	& 2	& 1	& $969434388$	& 6	& 10	& \hbox{\sout{$969430980$}} & 0 & 1	\\ \hline
% 8	& \hbox{\sout{$969435012$}}	& 0	& 7	& \hbox{\sout{$969434988$}}	& 4	& 5	& \hbox{\sout{$969431100$}}	& 1	& 0	\\ \hline
% 9	& $969435612$	& 5	& 2	& \hbox{\sout{$969435588$}}	& 2	& 0	& $969431220$	& 2	& 10	\\ \hline
% 10	& \hbox{\sout{$969436212$}}	& 3	& 8	& \hbox{\sout{$969436188$}}	& 0	& 6	& \hbox{\sout{$969431340$}}	& 3	& 9	\\ \hline
% \end{tabular}
% \end{table}
$n \mod h_2 = 2$ 
\begin{table}[H]
\centering
\caption{First 13 iterations of the example}
\label{table:b}
\begin{tabular}{|c|c|@{}c@{}|@{}c@{}|c|@{}c@{}|@{}c@{}|c|@{}c@{}|@{}c@{}|}
\hline
	& \multicolumn{3}{|c|}{$r_1(k) = 600k_1+12$} &  \multicolumn{3}{|c|}{$r_2(k) = 600k_2+588$} & \multicolumn{3}{|c|}{$r_3(k) = 120k_3+60$}	\\ \hline
i   & $r_1(k)$	& $\bmod h_1$	& $\bmod h_2$	& $r_2(k)$	& $\bmod h_1$	& $\bmod h_2$	& $r_2(k)$	& $\bmod h_1$	& $\bmod h_2$ \\ \hline
0	& $969430212$ & 2 & 10	& \sout{$969430212$} & 6 & \sout{12}	& \sout{$969430212$} & \sout{0} & 3	\\ \hline
1	& \sout{$969430812$} & \sout{0}	& \sout{12}	& \sout{$969430788 $} & \sout{4} & \sout{1}	& \sout{$969430260$} & 1 &	\sout{6}	\\ \hline
2	& \sout{$969431412$} & 5 & \sout{1}	& $969431388$	& 2	& 3	& $969430380$	& 2	& \sout{9}	\\ \hline
3	& \sout{$969432012$} & \sout{3} & 3	& \sout{$969431988$}	& \sout{0}	& 5	& \sout{$969430500$} & \sout{3} & \sout{12}	\\ \hline
4	& $969432612$ & 1 & 5	& \sout{$969432588$}	& 5	& \sout{7}	& \sout{$969430620$}	& \sout{4}	& 2	\\ \hline
5	& \sout{$969433212$} & \sout{6}	& \sout{7}	& \sout{$969433188$}	& \sout{3}	& \sout{9}	& $969430740$	& 5	& 5	\\ \hline
6	& \sout{$969433812$} & \sout{4}	& \sout{9}	& $969433788$	& 1	& 11	& $969430860$	& 6	& 8	\\ \hline
7	& $969434412$ & 2 & 11	& \sout{$969434388$}	& 6	& \sout{0}	& \sout{$969430980$} & \sout{0} & 11	\\ \hline
8	& \sout{$969435012$} & \sout{0} & \sout{0}	& \sout{$969434988$}	& \sout{4}	& 2	& \sout{$969431100$}	& 1	& \sout{1}	\\ \hline
9	& $969435612$ & 5 & 2	& \sout{$969435588$}	& 2	& \sout{4}	& \sout{$969431220$}	& 2	& \sout{4}	\\ \hline
10	& \sout{$969436212$} & \sout{3} & \sout{4}	& \sout{$969436188$}	& \sout{0}	& \sout{6}	& \sout{$969431340$}	& \sout{3}	& \sout{7}	\\ \hline
11	& \sout{$969436812$} & 1 & \sout{6}	& $969436788$	& 5	& 8	& \sout{$969431340$}	& \sout{4}	& 10	\\ \hline
12	& $969437412$ & 6 & 8	& \sout{$969437388$}	& \sout{3}	& 10	& \sout{$969431340$}	& 5	& \sout{0}	\\ \hline
\end{tabular}
\end{table}
% \begin{table}[H]
% \centering
% \caption{First 7 iterations of the example}
% \label{table:b}
% \begin{tabular}{|c|c|c|c|c|}
% \hline
% i	& $k_{1,i}$, $k_{2,i}$, $k_{3,i}$ & $r_1(k)$	& $r_2(k)$	& $r_3(k)$	\\ \hline
% 0	& $1615716,1615717,8078584$ & $969430212 \bmod h = 2$	& $969430188 \bmod h = 6$	& \hbox{\sout{$969430140 \bmod h = 0$}}	\\ \hline
% 1	& $1615717$, $1615718$, $8078585$ & \hbox{\sout{$969430812 \bmod h = 0$}}	& \hbox{\sout{$969430788 \bmod h = 4$}}	& $969430260 \bmod h = 1$	\\ \hline
% 2	& $1615718$, $1615719$, $8078586$ & $969431412 \bmod h = 5$	& $969431388 \bmod h = 2$	& $969430380 \bmod h = 2$	\\ \hline
% 3	& $1615719$, $1615720$, $8078587$ & \hbox{\sout{$969432012 \bmod h = 3$}}	& \hbox{\sout{$969431988 \bmod h = 0$}}	& \hbox{\sout{$969430500 \bmod h = 3$}}	\\ \hline
% 4	& $1615720$, $1615721$, $8078588$ & $969432612 \bmod h = 1$	& $969432588 \bmod h = 5$	& \hbox{\sout{$969430620 \bmod h = 4$}}	\\ \hline
% 5	& $1615721$, $1615722$, $8078589$ & $969433212 \bmod h = 6$	& \hbox{\sout{$969433188 \bmod h = 3$}}	& $969430740 \bmod h = 5$	\\ \hline
% 6	& $1615722$, $1615723$, $8078590$ & \hbox{\sout{$969433812 \bmod h = 4$}}	& $969433788 \bmod h = 1$	& $969430860 \bmod h = 6$	\\ \hline
% \end{tabular}
% \end{table}
% 	\item Discard all values $r(k+\gamma h)$ corresponding to the first h values of r(k) that is crossed as shown in the above table. Traverse the remaining space until a value of r(k) that satisfies the condition is reached. The value for 
%    $k$ where a solution exists is $k_{3,1747048} = 9825632$.\\
Number of tries $= \lceil \frac{4 \times 6}{7 \times 13}  5241144 \rceil$\\
% \end{enumerate}
% \end{itemize}
~\\
\begin{proposition}
Let $h_1,h_2,...h_d$ be distinct odd primes, and $x = \prod_{i=1}^{d-1} h_i$ then
$\mathbb{Z}_{h_d} = \{0,...,h_d-1\}$
and let $K = \{r(k+\gamma x) \bmod h_d \mid \gamma \in \mathbb{Z}_{h_d}\}$
then $K = \mathbb{Z}_{h_d}$\\
% Let $h$ be an odd prime $\geq 7$ and $x$ a multiple of an odd prime with $h \nmid x$ then,
% \begin{equation*}
% \mathbb{Z}_{h} = \{r(k) \bmod h, r(k+x), ..., r(k+(h-1)x) \bmod h\}
% \end{equation*}
\end{proposition}
\begin{proof}
% \hfill \newline
% \noindent
$$ r(k+ \gamma_1 x) = r(k+ \gamma_2 x) \bmod h $$
$$ a(k+ \gamma_1 x) + b = a(k+ \gamma_2 x) + b \pmod h $$
$$ a(k+ \gamma_1 x) = a(k+ \gamma_2 x) \bmod h $$
$$ k+ \gamma_1 x = k+ \gamma_2 x \pmod h $$
$$ \gamma_1 x = \gamma_2 x \mod h $$
$$ \gamma_1 = \gamma_2 \mod h $$
\end{proof}
}
% 		\subsubsection{Product of Primes}
% 	\subsection{Factoring}
	\subsection{Complexity Analysis}
		For $b = \log_2(p-q)$\\
		in the worst case $b \approx \log_2(n)-1$\\
		The factoring algorithm is split into two main parts a setup and search.
		\subsubsection{Complexity of The Setup Part}
			Given $h_1,...,h_i$ primes $\geq 7$ and their corrosponding possible $r \bmod h_i$ values. There are on average $\approx \frac{h_i}{2}$  $r_{j,i}$ values for any $ x_i = n \bmod h_i$.
			The complexity of a single 'mod' operation is roughly of the form $O(b\log_2(h_i))$. For each $h_i$ the mod operation is computed $h_i$ times.
			In total there are $\sum_i h_i$ mod operations in the setup phase.
			So, the complexity of the setup part is:\\
			$$O(\sum_i h_i b \log_2 h_i)$$
			Assumming $i=b$\\
			$$O(\sum_{i=1}^b h_i b \log_2 h_b)$$
		\subsubsection{Complexity of The Search Part}
			Initially the search part without the setup part made a condition check for each iteration and asumming each condition check necessitates a square root computation.
			$O[(2b)^2]$
			condtion checks in the search part are made $c(p-q)$ times.
			$$O[2^b (2b)^2]$$
			After adding the setup phase the nubmer of square root computations becomes $\approx \frac{c(p-q)}{2}$
			$$O( \frac{2^b }{2^i } (2b)^2 )$$
			assuming $ i= b$ is used
			$O((2b)^2)$
			The total expression:
			$$O(\sum_i h_i b \log_2 h_i) + O (\frac{2^b }{2^i } (2b)^2 ) )$$
			for $i = b$
			$$O(\sum_{i=0}^b b log_2 h_b +( (2b)^2 ))$$
			$$O((b \log_2(h_b)^2 + 4 b^2))$$
			$$O(b^2 (4+\log_2(h_b)^2))$$

% %%%%%%%%%%%%%%%%%%%%
% \textcolor{red}{
% \subsection{Finding a formula for $p+q$}
% }
% %%%%%%%%%%%%%%%%%%%%
% \textcolor{red}{
% Consider a prime not equal to $2,3,5$, then it has one of the forms:
% }
% \begin{enumerate}
%     \item $10x+1$
%     \item $10x+3$
%     \item $10x+7$
%     \item $10x+9$
% \end{enumerate}
% \textcolor{red}{
% Let $p=10x+z_1$, and without loss of generality, let $q=10(x+j)+z_2$ where $z_1, z_2=1,3,7,9$ and $j \geq 0$ if $z_1 \neq z_2$ or $j > 0$ if $z_1 = z_2$. Then $n=pq$ can have one of the forms:
% }
% \begin{enumerate}\setlength\itemsep{1em}
%     \item $100x(x+j)+10(2x+j)+1$
%     \item $100x(x+j+1)+10(8x+9j+8)+1$
%     \item $100x(x+j+1)+10(7j+2)+1$
%     \item $100x(x+j+1)+10(3j+2)+1$
%     \item $100x(x+j)+10(4x+3j)+3$
%     \item $100x(x+j)+10(4x+j)+3$
%     \item $100x(x+j+1)+10(6x+9j+6)+3$
%     \item $100x(x+j+1)+10(6x+7j+6)+3$
%     \item $100x(x+j)+10(8x+7j)+7$
%     \item $100x(x+j)+10(8x+j)+7$
%     \item $100x(x+j+1)+10(2x+9j+2)+7$
%     \item $100x(x+j+1)+10(2x+3j+2)+7$
%     \item $100x(x+j)+10(6x+3j)+9$
%     \item $100x(x+j+1)+10(4x+7j+4)+9$
%     \item $100x(x+j+1)+10(9j)+9$
%     \item $100x(x+j+1)+10j+9$
% \end{enumerate}
% and $r=p+q$ will have one of the following forms respectively:
% \begin{enumerate}\setlength\itemsep{1em}
%     \item $10(2x+j)+2$
%     \item $10(2x+j+1)+8$
%     \item $10(2x+j+1)$
%     \item $10(2x+j+1)$
%     \item $10(2x+j)+4$
%     \item $10(2x+j)+4$
%     \item $10(2x+j+1)+6$
%     \item $10(2x+j+1)+6$
%     \item $10(2x+j)+8$
%     \item $10(2x+j)+8$
%     \item $10(2x+j+1)+2$
%     \item $10(2x+j+1)+2$
%     \item $10(2x+j)+6$
%     \item $10(2x+j+1)+4$
%     \item $10(2x+j+1)$
%     \item $10(2x+j+1)$
% \end{enumerate}
% From the above two lists, if $n=1 \mod 10$, then $r \in \{x \mid x=0,2,8 \mod 10\}$. As we can see, $r$ now is in a smaller set than the set of all even positive integers. 

% We want to further reduce the set that contains $r$. By letting $n=11 \mod 100$, we can use the first formula in each list:          
% \begin{align*}
% 0  n &= 100x(x + j) + 10(2x + j) + 1 \\
%   r &= 10(2x + j) + 2 
% \end{align*}
% \textcolor{red}{
% Notice that $j$ is odd since $2x+j = 1 \mod 10$. We also notice that $4 \mid n+1$ since $n+1 = 2 \mod 10$ and $2x+j$ is odd (e.g. 12, 32, ...) Therefore, by proposition 2, we conclude that $4 \mid r$. 
% }

% Next, we consider the divisibility by $8$ and arrive to the following statement:

% \begin{align*}
%   4k &= 8 \mod 10 \\
%   &\implies 2k = 4 \mod 5 \\
%   &\implies k = 2 \mod 5 \\
%   &\implies k = 5m + 2
% \end{align*}
% so $2x + j = 4k + 3 = 4(5m + 2)+ 3 = 20m + 11$, then:
% \textcolor{red}{
% \begin{align*}
%   n &= 100x(x + j) + 10(2x + j) + 1 \\
%     &= 100x(x + j) + 10(20m + 11) + 1 \\
%     &= 100x(x + j) + 200m + 110 + 1 \\
%     &= 100[x(x + j) + 2m + 1] + 11
% \end{align*}
% %}
% %\textcolor{red}{
% We notice in the above expression that $2m + 1$ is always odd, and as we previously noted that $j$ is odd, then $x(x+j)$ will always be even regardless of $x$. Consequently, the expression $x(x+j) + 2m + 1$ is always odd. Therefore, if $8 \mid r$, then the hundreds is odd.
% }

% \noindent
% $(\implies)$ Suppose $8 \nmid r$, then $2x + j = 4k + 1$, so
% \begin{align*}
%   &2x + j = 4k + 1 = 1  \mod 10 \\
%   &\implies 4k = 0      \mod 10 \\
%   &\implies 2k = 0      \mod 5 \\
%   &\implies  k = 0      \mod 5 \\
%   &\implies  k = 5m
% \end{align*}
% so $2x + j = 4k + 1 = 4(5m) + 1 = 20m + 1$, then:
% \textcolor{red}{
% \begin{align*}
%   n &= 100x(x + j) + 10(2x + j) + 1 \\
%     &= 100x(x + j) + 10(20m + 1) + 1 \\
%     &= 100x(x + j) + 200m + 10 + 1 \\
%     &= 100[x(x + j) + 2m] + 11
% \end{align*}
% %}
% %\textcolor{red}{
% We notice in this case that $2m$ is even and, as explained above, $x(x+j)$ will always be even regardless of $x$, so the expression $x(x+j) + 2m$ is always even. Therefore, if $8 \nmid r$, then the hundreds is even.
% }

% For the first formula in the list, let $n = 11 \mod 100$, by combining the two results, we prove statement~\ref{prop3}.

% Let $n=11 \mod 100$. If the hundreds is odd, then one of the possible formulas:
% \[r=10(20m+11)+2=200m+112\]
% and if the hundreds is even, then one of the possible formulas:
% \[r=10(20m+1)+2=200m+12\]
% By combining this result with proposition 1, we get:
% \begin{table}[H]
% \centering
% \begin{tabular}{|l|l|l|l|}
% \hline
% \multicolumn{2}{|l|}{Hundreds odd} & \multicolumn{2}{l|}{Hundreds even} \\ \hline
%  $3 \mid n+1$ & $3 \nmid n+1$  & $3 \mid n+1$ & $3 \nmid n+1$  \\ \hline
% $r=600k+312$ & $r=600k+112$ & $r=600k+12$  & $r=600k+212$ \\ \hline
%              & $r=600k+512$ &               & $r=600k+412$ \\ \hline
% \end{tabular}
% \end{table}

% The above table is for the first form in the list of $n$ \textcolor{red}{when $n=11 \mod 100$}. The rest of the forms are provided in the Appendix.

% \subsection{Example}
% \textcolor{red}{
% In this section, we give an example on how we can find $r=p+q$ given $n$, and subsequently $p$ and $q$ using equation \ref{eq:1}. Notice that $p+q > 2\sqrt{n}$ which we use to further improve our search.
% }

% We start by looking at the value $n$ and find the table of formulas corresponding to the form of $n$. After that, we use the formulas to search for $r$. We find $r$ when the value $\sqrt{r^2 - 4n}$ is equal to an integer (see Appendix A).

% Given $n=9476465591$, we will try to find $p$ and $q$ (we computed $n$ from $p=101693$ and $q=93187$ so we are going to check that our solution matches these values).
% \begin{enumerate}\setlength\itemsep{1em}
%     \item $n=91 \mod 100$, so we look at table~\ref{table:10} in Appendix B.
%     \item $3 \mid n+1$ and the hundreds is odd, so $r$ could be one of the formulas: \\
%     $r_1 = 600k_1+192$ \\
%     $r_2 = 600k_2+408$ \\ 
%     $r_3 = 120k_3$
%     \item $2\sqrt{n} \approx 194694$. Since $r_1, r_2, r_3 > 2\sqrt{n}$, the starting points will be: \\
%     $k_{1,0} = 324$ \\
%     $k_{2,0} = 324$ \\ 
%     $k_{3,0} = 1623$ 
%     \item We evaluate $r_1, r_2, r_3$ starting with the above values for $k_{1,0}, k_{2,0}, k_{3,0}$ respectively, and continue to increment the $k$'s until $\sqrt{r^2 - 4n}$ is equal to an integer. We find the right value for $k$ to be: \\ 
%     $k_{3,1} = k_{3,0} + 1 = 1624$ \\
%     Now we compute $r$ in the following way: \\
%     $r = 120(k_{3,1}) = 120(1624)=194880$
%     \item $p-q = \sqrt{r^2-4n} = \sqrt{37978214400-37905862364} = 8506$
%     \item $p = {[r+(p-q)]/2} = 203386/2 = 101693$ \\ 
%     $q = n/p = 9476465591/101693 = 93187$
% \end{enumerate}
% (Notice in this example that $k$ is small; this is due to $p-q$ being small.)


\section{Hello this is section }
testj jthe rest of the document l
testj jthe rest of the document l
testj jthe rest of the document l
testj jthe rest of the document l
testj jthe rest of the document l
testj jthe rest of the document l
testj jthe rest of the document l
testj jthe rest of the document l
testj jthe rest of the document l
testj jthe rest of the document l
testj jthe rest of the document l
testj jthe rest of the document l
testj jthe rest of the document l
testj jthe rest of the document l
testj jthe rest of the document l
testj jthe rest of the document l
testj jthe rest of the document l
testj jthe rest of the document l
testj jthe rest of the document l
testj jthe rest of the document l
testj jthe rest of the document l
testj jthe rest of the document l
testj jthe rest of the document l
testj jthe rest of the document l
testj jthe rest of the document l
testj jthe rest of the document l
testj jthe rest of the document l
testj jthe rest of the document l
\end{document}
